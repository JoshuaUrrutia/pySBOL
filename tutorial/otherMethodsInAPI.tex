\section*{Other Features of {\tt pySBOL}}
So far, we have demonstrated how one can build the CRISPR-based repression module ~\cite{kiani2014crispr} using {\tt pySBOL}. In this section, we present other major methods in the library's API. 

\subsection*{Retrieving an Existing Object}
Often, we need getter methods to retrieve a previously created object. You can easily retrieve top level objects from a document by calling a templated ``get'' method using the class of the target object as the template argument. For example, if we want to get the \lstinline+cas9_generic+ protein \sbol{ComponentDefinition} object, we can use the \lstinline+get<ComponentDefinition>+ method shown below (lines 1-4) by providing the display ID of the object. By default this retrieves the latest version of an object. Alternatively, one may pass a full URI as an argument to the getter, which may be necessary when retrieving previous versions of an object.

\vspace{\abovedisplayskip}
\begin{minipage}{0.95\textwidth} 
\begin{lstlisting}
cas9_generic2 = doc.getComponentDefinition("cas9_generic")
\end{lstlisting}
\end{minipage}

\subsection*{Manipulating Optional Fields}
Objects may include optional fields. These are indicated in the UML specification as properties having 0 or more possible values. For example, the role property of a ComponentDefinition is optional while the molecular type field is required. Optional properties can only be set after the object is created. The following code creates a DNA component which is designated as a promoter:

\vspace{\abovedisplayskip}
\begin{minipage}{0.95\textwidth} 
\begin{lstlisting}
TargetPromoter = ComponentDefinition("TargetPromoter", BIOPAX_DNA, "1.0.0")
TargetPromoter.roles.set(SO_PROMOTER)
\end{lstlisting}
\end{minipage}

In addition, properties have a get method. To view the value of a property:

\vspace{\abovedisplayskip}
\begin{minipage}{0.95\textwidth} 
\begin{lstlisting}
print(TargetPromoter.roles.get())
# This returns the string "http://identifiers.org/so/SO:0000167" which is the Sequence Ontology term for a promoter.
\end{lstlisting}
\end{minipage}

Note also that some properties may contain more than one value. In the specification diagrams, an asterisk symbol next to a property indicates that the property may hold an arbitrary number of values. For example, a ComponentDefinition may be assigned multiple roles. To append a new value to the values already assigned:

\vspace{\abovedisplayskip}
\begin{minipage}{0.95\textwidth} 
\begin{lstlisting}
TargetPromoter.roles.add(SO + "0000568")
\end{lstlisting}
\end{minipage}

To get multiple values back from a property, it is necessary to iterate over the property:

\vspace{\abovedisplayskip}
\begin{minipage}{0.95\textwidth} 
\begin{lstlisting}
# Iterate through a property to get multiple values
for i in range(len(reaction_participant.roles.get())):
    role = reaction_participant.roles[i]
    print(role)
\end{lstlisting}
\end{minipage}

An important thing to remember is that the set method will always overwrite the first value of a property, while the add method will always append a new value. To remove a value, one may use the remove method. Currently the remove method requires a numerical index, though this will likely change in the future.

\vspace{\abovedisplayskip}
\begin{minipage}{0.95\textwidth} 
\begin{lstlisting}
TargetPromoter.roles.remove(0)
\end{lstlisting}
\end{minipage}

The only exceptions where these methods are not available are the following three fields in the \lstinline+Identified+ class: \lstinline+persistentIdentity+, \lstinline+displayId+, and
\lstinline+version+.  These fields cannot be edited, since they are crucial to maintaining
compliant SBOL objects (see Section~11.2 ``Compliant SBOL Objects'' of the~\href{http://sbolstandard.org/downloads/specification-data-model-2-0/}{Specification
  (Data Model 2.0)} for more details).

\subsection*{Creating and Editing References}

Some SBOL objects point to other objects by way of references. For example, ComponentDefinitions point to their corresponding Sequences. Properties of this type should be set with the URI of the related object.

\vspace{\abovedisplayskip}
\begin{minipage}{0.95\textwidth} 
\begin{lstlisting}
EYFPGene = ComponentDefinition("EYFPGene", BIOPAX_DNA)
seq = Sequence("EYFPSequence", "atgnnntaa", SBOL_ENCODING_IUPAC)
EYFPGene.sequences.set(seq.identity.get())
\end{lstlisting}
\end{minipage}

\subsection*{Creating and Editing Child Objects}
Some SBOL objects can be composed into hierarchical parent-child relationships. In the specification diagrams, these relationships are indicated by black diamond arrows. For example ComponentDefinitions are parents of SequenceAnnotations. 

If operating in SBOL-compliant mode, you will almost always want to use the create method rather than constructors in order to create a child object. The create method constructs and adds the SequenceAnnotation in a single function call. The create method ALWAYS takes one argument–-the displayId of the new object. Some values may be initialized with default values. Refer to documentation of specific constructors to learn which parameters are assigned default values. After object creation, these fields and optional fields may be changed. 

\vspace{\abovedisplayskip}
\begin{minipage}{0.95\textwidth} 
\begin{lstlisting}
point_mutation = TargetPromoter.sequenceAnnotations.create("point_mutation")
\end{lstlisting}
\end{minipage}

In SBOL-compliant mode, directly adding a child to a parent object is prohibited, in order to maintain URI persistence between them. In `open-world mode' the library makes no assumptions about how URIs are formed and leaves URI generation entirely up to the user.  In this case child objects can be directly created using constructors and added to the parent. Use \lstinline+toggleSBOLCompliance()+ if you prefer to generate your own URIs and operate in open-world mode. In future developments, constructors may be opened up for use in SBOL-compliant mode as well.

\vspace{\abovedisplayskip}
\begin{minipage}{0.95\textwidth} 
\begin{lstlisting}
point_mutation = SequenceAnnotation("point_mutation")
TargetPromoter.sequenceAnnotations.add(point_mutation)
\end{lstlisting}
\end{minipage}

\subsection*{Serialization}
The library supports reading and writing data encoded in RDF/XML format. All file I/O operations are performed on the Document object. The read and write methods are used for reading and writing files in SBOL format.

\vspace{\abovedisplayskip}
\begin{minipage}{0.95\textwidth} 
\begin{lstlisting}
doc = Document()
doc.read("CRISPR_example.xml")
doc.write("CRISPR_example_new.xml")
\end{lstlisting}
\end{minipage}

The complete repression model described in this tutorial is provided in the libSBOL source code in the examples directory. This example is self-contained in that you can run it to generate the RDF/XML output. Note that SBOL does not provide the specification of a mathematical model directly. It is possible, however, to generate a mathematical model using SBML~\cite{SBML} and the procedure described in~\cite{roehner2015generating}. Then, the SBOL document can reference this generated SBML model.

\subsection*{Validation}
The library also supports validation of RDF/XML file to ensure that it conforms with SBOL specification. Validation is performed on a Document object over online validator. To run it, simply run validate() on a Document object. The returned string will contain the results of validation.

\vspace{\abovedisplayskip}
\begin{minipage}{0.95\textwidth} 
\begin{lstlisting}
>>> result = doc.validate()
>>> print(result)
Validation successful, no errors.
\end{lstlisting}
\end{minipage}

Validation is also run when a SBOL file is created through \lstinline+write()+ function. The output of validation is returned as a string when Document.write() function is executed. Keep in mind that the file will be generated regardless of whether it passes the validation step or not.

\vspace{\abovedisplayskip}
\begin{minipage}{0.95\textwidth} 
\begin{lstlisting}
>>> result = doc.write("CRISPR_example.xml")
>>> print(result)
Validation successful, no errors.
\end{lstlisting}
\end{minipage}


% Need to verify
% \subsection*{Copying Objects}
% The library can make copies of \sbol{TopLevel} objects using the templated \lstinline+copy+ methods.  This method takes a number of optional arguments. If no arguments are specified, a copy is made and the version is incremented. An object can be copied from one Document to another by passing a pointer to the target Document as the first argument. In addition, the object can be copied to a new namespace, which is specified as the optional second argument. Finally, a custom version tag can be specified in the third argument. The following code snippet demonstrates how the copy method may be used to copy a ComponentDefinition to a new namespace and Document.

% \vspace{\abovedisplayskip}
% \begin{minipage}{0.95\textwidth} 
% \begin{lstlisting}
% venus = old_doc.getComponentDefinition('venus_yfp')
% venus_copy = venus.components.copy()
% \end{lstlisting}
% \end{minipage}